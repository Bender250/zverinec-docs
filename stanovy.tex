\documentclass[11pt,a4paper]{article}

% ===== LOADING PACKAGES =====
% language settings, main documnet language last
\usepackage[czech]{babel}
% enabling new fonts support (nicer)
\usepackage{lmodern}
% setting input encoding
\usepackage[utf8]{inputenc}
% setting output encoding
\usepackage[T1]{fontenc}
% set page margins
\usepackage[top=2cm, bottom=2cm, left=2cm, right=2.5cm]{geometry}	% pc version
% package to make bullet list nicer
\usepackage{enumitem}

% change section to be in format 'Článek I. XYZ'
\renewcommand{\thesection}{Článek \Roman{section}.} 

\begin{document}
\section*{\Huge{}Stanovy spolku}

Úplné znění ke dni 9.\ 12.\ 2014.

\section{Úvodní ustanovení}
\begin{enumerate}[itemsep=0pt]
    \item Název spolku: Spolek přátel severské zvěře (dále též „Spolek“)
    % Pokud bychom chtěli právnickou formu, pak je potřeba se zde přihlásit, 
    % že jsme spolkem podle NOK.
    % „XYZ je spolek ve smyslu ustanovení § 214 a násl. zák. č. 89/2012 Sb., a 
    % jako takový je právnickou osobou způsobilou k právnímu jednání.“
    \item Sídlo Spolku: Botanická 68a, 602 00 Brno
    \item Spolek je samosprávná, dobrovolná, nepolitická a nezisková organizace 
    založená za účelem naplňování společného zájmu, kterým je poskytování obecně
    prospěšných činností v oblastech vzdělání, propagace vědy, zejména 
    informatiky a dalších přírodních věd. Posláním Spolku je zejména: 
    \begin{enumerate}[itemsep=0pt,topsep=0pt]
        \item propagovat přírodní vědy a racionální přístupu k řešení problémů,
        \item propagovat netradiční formy vzdělávání a výuky,
        \item propagovat Fakultu informatiky Masarykovy univerzity a Masarykovu univerzitu obecně.
    \end{enumerate}
    \item Vnitřní organizace Spolku, práva a povinnosti členů i volených orgánů 
    Spolku se řídí těmito stanovami, které jsou uloženy ve svém úplném znění 
    v sídle Spolku a dostupné na webových stránkách Spolku.
\end{enumerate}

\section{Činnost Spolku}
\begin{enumerate}[itemsep=0pt]
    \item Hlavní činnost Spolku směřuje k ochraně a uspokojení společných zájmů 
    a naplňování poslání, tak jak jsou stanoveny v článku I. To je prováděno 
    zvláště prostřednictvím: 
    \begin{enumerate}[itemsep=0pt,topsep=0pt]
        \item organizování zážitkových a vzdělávacích programů,
        \item organizování netradičních soutěží a her,
        \item pořádání přednášek a dalších propagačních aktivit.
    \end{enumerate}
    \item K podpoře hlavní činnosti vyvíjí Spolek dále činnosti vedlejší: 
    \begin{enumerate}[itemsep=0pt,topsep=0pt]
        \item organizace akcí pro členy Spolku pro zajištění kontinuity znalostí
         a zkušeností potřebných pro činnosti Spolku,
        \item zlepšování prostředí Fakulty informatiky Masarykovy univerzity a Masarykovy univerzity obecně,
        \item spolupráce s jinými organizacemi při naplňování hlavní činnosti 
        Spolku,
        \item podpora organizací se stejným nebo podobným cílem, jako je 
        poslání Spolku,
        \item vytváření společenské, ekonomické a materiálně technické 
        základny pro naplňování poslání Spolku.
    \end{enumerate}
\end{enumerate}

\section{Členství}
\begin{enumerate}[itemsep=0pt]
    \item Členství ve Spolku je dobrovolné.
    \item Rozlišuje se dvojí členství:
    \begin{enumerate}[itemsep=0pt,topsep=0pt]
        \item sympatizující členství,
        \item aktivní členství.
    \end{enumerate}
    \item Sympatizujícím i aktivním členem Spolku se může stát každá fyzická 
    osoba starší 18 let, bez rozdílu pohlaví, vyznání, politického a sociálního 
    zařazení, národnosti, rasy a státní příslušnosti. Členství se váže na osobu 
    člena, je nepřevoditelné na jinou osobu a nepřechází na jeho 
    právního nástupce. 
    \item Fyzická osoba se stává sympatizujícím nebo aktivním členem ke dni 
    rozhodnutí výboru Spolku o přijetí její žádosti o členství. Výbor Spolku, 
    může odmítnout přijetí žadatele o členství, pokud dojde k většinovému 
    názoru, že přijetím žadatele za člena by bylo nebo mohlo být ohroženo 
    řádné plnění poslání Spolku, případně že žadatel nejeví dostatečný zájem 
    o naplňování poslání Spolku.
    \item Ukončení členství:
    \begin{enumerate}[itemsep=0pt,topsep=0pt]
        \item dobrovolné vystoupení člena – členství končí dnem doručení 
        písemného oznámení o ukončení členství výboru,
        \item úmrtím člena,
        \item vyloučením člena – členství končí dnem doručení rozhodnutí 
        o vyloučení člena ze Spolku; výbor Spolku má právo vyloučit člena, 
        pokud svým jednáním porušuje zásady a cíle Spolku, zanedbává členské 
        povinnosti, ohrožuje řádně plnění poslání Spolku nebo se již déle než 
        rok nepodílí na naplňování poslání Spolku.
    \end{enumerate}
    \item Právo sympatizujících i aktivních členů:
    \begin{enumerate}[itemsep=0pt,topsep=0pt]
        \item účastnit se činnosti Spolku,
        \item být pravidelně informován o dění ve Spolku,
        \item podílet se na stanovování náplně a forem činnosti Spolku,
        \item podávat návrhy, připomínky, vznášet dotazy orgánům Spolku.
    \end{enumerate}
    \item Práva aktivních členů:
    \begin{enumerate}[itemsep=0pt,topsep=0pt]
        \item užívat výhod aktivního člena Spolku, pokud jsou takové vyhlášeny,
        \item volit a být volen do řídících a kontrolních orgánů Spolku.
    \end{enumerate}
    \item Povinnosti sympatizujících i aktivních členů:
    \begin{enumerate}[itemsep=0pt,topsep=0pt]
        \item chránit a zachovávat dobré jméno Spolku,
        \item dodržovat stanovy Spolku,
        \item přispívat svou činností k poslání Spolku,
        \item dodržovat a naplňovat Společně dohodnuté postupy a zásady.
    \end{enumerate}
    \item Výbor Spolku je povinen vést seznam sympatizujících i aktivních členů,
    který je veřejně dostupný na webové stránce Spolku. 
\end{enumerate}

\section{Orgány Spolku}
\begin{enumerate}[itemsep=0pt]
    \item  Pro zabezpečení činnosti Spolku jsou zřízeny následující orgány: 
    % Pokud bychom chtěli být spolkem podle NOZ, pak zde musí být ještě 
    % kontrolní orgán, např: kontrolní komise.
    % Také je potřeba přidat veškeré věci s finančním hospodařením atp.
    \begin{itemize}[itemsep=0pt,topsep=0pt]
        \item členská schůze,
        \item výbor,
        \item předseda.
    \end{itemize}
    \item Práva a povinnosti jednotlivých orgánů, způsob vzniku a zániku jejich 
    funkce a průběh jejich jednání je upraven dále ve stanovách.
    \item Členská schůze je tvořena shromážděním všech členů Spolku, ostatní 
    orgány jsou volené.
    \item Funkční období volených orgánů je šest měsíců. Členové volených 
    orgánů Spolku, jejichž počet neklesl pod polovinu, mohou kooptovat náhradní 
    členy svého orgánu do nejbližšího zasedání členské schůze. Členství 
    ve volených orgánech zaniká (kromě smrti člena voleného orgánu) uplynutím 
    funkčního období, nebo odvoláním člena voleného orgánu členskou schůzí 
    pro hrubé porušení jeho povinností.
\end{enumerate}

\section{Členská schůze}
\begin{enumerate}[itemsep=0pt]
    \item Členská schůze je tvořena shromážděním všech členů Spolku a je 
    nejvyšším orgánem Spolku. Členská schůze projednává činnost Spolku 
    za uplynulé období, přijímá zásady činnosti pro období následující, 
    volí volené orgány Spolku, hodnotí práci odstupujících orgánů a přijímá 
    další rozhodnutí zásadní důležitosti pro existenci a činnost Spolku. 
    Do její působnosti tak náleží: 
    \begin{enumerate}[itemsep=0pt,topsep=0pt]
        \item určovat poslání, hlavní i vedlejší zaměření činnosti Spolku,
        \item rozhodovat o změně stanov,
        \item volit členy výboru a předsedu,
        \item hodnotit činnost dalších orgánů Spolku i jejich členů,
        \item rozhodnout o dobrovolném rozpuštění Spolku.
    \end{enumerate}
    \item Členská schůze je svolávána výborem Spolku podle potřeby, nejméně 
    však jednou za šest měsíců.
    \item Na žádost nejméně jedné třetiny členů musí být svolána mimořádná 
    členská schůze. Výbor je povinen svolat mimořádnou členskou schůzi 
    nejpozději do čtyř týdnů od doručení žádosti, která musí obsahovat uvedení 
    důvodu a program mimořádné členské schůze.
    \item Právo zúčastnit se členské schůze, právo hlasovat, volit a být volen 
    mají všichni aktivní členové Spolku. Každý sympatizující nebo aktivní člen 
    je oprávněn účastnit se zasedání a požadovat i dostat na něm vysvětlení 
    záležitostí Spolku, vztahuje-li se požadované vysvětlení k předmětu zasedání
    členské schůze.
    \item Každý aktivní člen má jeden hlas, hlasy všech aktivních členů mají 
    stejnou váhu. Členská schůze je schopna usnášet se za účasti nadpoloviční 
    většiny všech aktivních členů Spolku. Usnesení členská schůze přijímá 
    většinou hlasů přítomných aktivních členů. 
    \item Jednání členské schůze řídí předseda, nebo jím pověřená osoba.
    \item Výbor zajistí vyhotovení zápisu ze zasedání do čtrnácti dnů od jejího 
    ukončení. Není-li to možné, vyhotoví zápis ten, kdo zasedání předsedal nebo 
    koho tím pověřila členská schůze. Ze zápisu musí být patrné, kdo zasedání 
    svolal a jak, kdy se konalo, kdo je zahájil, kdo mu předsedal, jaká usnesení
    byla přijata a kdy a kým byl zápis vyhotoven. Zápisy jsou dostupné veřejně 
    na webové stránce Spolku.
\end{enumerate}

\section{Výbor}
\begin{enumerate}[itemsep=0pt]
    \item Výbor je druhým nejvyšším orgánem Spolku, řídí jeho činnost v souladu 
    se stanovami po celé své funkční období. 
    \item Výbor je statutárním orgánem a má 5 členů. Jednat jménem Spolku může 
    předseda společně s dalším členem výboru nebo libovolní tři členové výboru.
    \item Výbor odpovídá za řádný provoz Spolku, dohlíží na dodržování pravidel 
    a zejména dbá na řádné naplňování poslání Spolku.
    \item Výbor vede a řídí práci Spolku v období mezi členskými schůzemi, 
    dohlíží na dodržování stanov, pečuje o jeho rozvoj.
    \item Výbor je oprávněn delegovat ty své pravomoci, o kterých rozhodne na 
    další členy. Výbor je povinen jednat s péčí řádného hospodáře. 
    \item Výbor je svoláván libovolným členem výboru podle potřeby. Výbor je 
    usnášeníschopný pouze za účasti tří pětin jeho členů. Hlasy všech členů 
    výboru mají stejnou váhu.
    \item Do výlučné kompetence výboru patří: 
    \begin{enumerate}[itemsep=0pt,topsep=0pt]
        \item svolávat členskou schůzi, 
        \item posuzovat a rozhodovat o návrzích jednotlivých aktivit 
        realizovaných Spolkem a to včetně delegace svých pravomocí potřebných 
        pro realizaci těchto aktivit na další členy Spolku,
        \item přijímat sympatizující i aktivní členy, rozhodovat o jejich 
        vyloučení,
        \item schvalovat interní organizační pravidla a zásady Spolku.
    \end{enumerate}
\end{enumerate}

\section{Předseda}
\begin{enumerate}[itemsep=0pt]
    \item Předseda je nejvyšším výkonným představitelem Spolku a navenek 
    za Spolek jedná společně s jedním z dalších členů výboru.  
    \item Předseda je volen členskou schůzí ze zvolených členů výboru. 
    Jeho funkční období končí předáním funkce nastupujícímu předsedovi. 
    Předání je nutné provést nejpozději do 14 dnů od zvolení nového předsedy. 
    \item K výlučným kompetencím předsedy patří: vedení členské schůze, 
    ověřování její usnášeníschopnosti, pořízení zápisu z jejího jednání.
    \item Předseda může delegovat některé své kompetence na další členy výboru, 
    případně na další členy Spolku.
\end{enumerate}

\section{Závěrečná ustanovení}
\begin{enumerate}[itemsep=0pt]
    \item Spolek může zaniknout dobrovolným rozpuštěním na základě rozhodnutí
    členské schůze.
    \item Dojde-li mezi členy Spolku ke sporu o výklad těchto stanov, pak je 
    k jejich výkladu oprávněna členská schůze.
    \item Spolek vzniká dnem přijetím těchto stanov členskou schůzí a zanesením 
    do seznamu spolků Masarykovy univerzity.
\end{enumerate}

\end{document}
