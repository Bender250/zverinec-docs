\documentclass[11pt,a4paper]{article}

% ===== LOADING PACKAGES =====
% language settings, main documnet language last
\usepackage[slovak]{babel}
% enabling new fonts support (nicer)
\usepackage{lmodern}
% setting input encoding
\usepackage[utf8]{inputenc}
% setting output encoding
\usepackage[T1]{fontenc}
% set page margins
\usepackage[top=2cm, bottom=2cm, left=2cm, right=2cm]{geometry}
% package to make bullet list nicer
\usepackage{enumitem}
% setting custom colors for links
\usepackage{xcolor}
\definecolor{dark-red}{rgb}{0.6,0.15,0.15}
\definecolor{dark-green}{rgb}{0.15,0.4,0.15}
\definecolor{medium-blue}{rgb}{0,0,0.5}
% generating hyperlinks in document
\usepackage{url}
\usepackage[plainpages=false, 	% get the page numbering correctly
            pdfpagelabels, 		% write arabic labels to all pages
            unicode,	 			% allow unicode characters in links
            colorlinks=true, 	% use colored links instead of boxed
            linkcolor={dark-red},
            citecolor={dark-green},
            urlcolor={medium-blue}
			]{hyperref}

\begin{document}
\section*{Zápis zo stretnutia Spolku přátel severské zvěře}
\textbf{Dátum:} 9.\ 12.\ 2014\\
\textbf{Prítomní:} Jan Drábek, Jan Mrázek, Martin Ukrop, Vladimír Štill, Martina Krasnayová, Martin Hanžl, Ondřej Slámečka, Henrich Lauko, Karel Kubíček, Tomáš Effenberger, Jiří Novotný, Jiří Mauritz, Vladimír Sedláček

\subsection*{Realizované akcie}
\begin{itemize}[itemsep=0pt]
\item Zhodli sme sa, že uskutočnené akcie (InterSoB, FIORD) dopadli veľmi dobre. Pozreli sme videá a fotografie z akcií.
\item Maara, Karel, Jirka a Martin boli ocenení Rádom priateľa severskej zvere za značnú pomoc s organizáciou FIORDu a InterSoBa.
\item Brněnská televízia uverejní krátky (15-20 min.) rozhovor s dvoma členmi spolku (zaujala ich reportáž o FIORDe v Lemurovi).
\end{itemize}

\subsection*{Plánované akcie spolku}

Prebehol brainstroming možných budúcich akcií spolku.
\begin{itemize}[itemsep=0pt]
\item \textbf{Náboj} -- myšlienka rozšíriť súťaž (matematický) Náboj aj do Brna. Viac inform=acií o súťaži na \url{http://math.naboj.org/index.php}.
\item \textbf{MiniFIORD} -- pravidelné malé FIORDy, buď v blízkosti fakulty alebo aj inde. Bitky v átriu, v centrálnom parku na prírodovede, v Lužánkách.
\item \textbf{Stanoviská InterSoBa} -- znovupoužitie existujúcich stanovísk. Voľne umiestnené v átriu alebo foyer fakulty, možnosť použi pri DoD, akciách pre olympiády, ...
\item \textbf{Schrödingerov gril} -- \uv{pokračovanie} fakultnej grilovačky v átriu. Keďže s grilom by mohol byť problém, možnosť zmeniť na piknik/pomazánkový deň. Zvážiť účasť zamestnancov, pozvať aspoň vybrané laby?
\item \textbf{InterJeLen} -- pokračovanie grilovačky organizátorov univerzitných akcií (nielen) pre stredoškolákov. Pozvať aj semináre z prírodovedy.
\item \textbf{Záhadné objekty na FI} -- drobná šifrovačka v budove FI bez oficiálneho uvedenia. Pre všímavých, ktorým sa páčia šifry.
\item \textbf{PS šalina} -- využitie možnosti DPMB umiestňovať plagáty do vybranej šaliny. Asembler v šaline? Krása matematiky v šaline? Šifra v šaline? Viac informácií napríklad \url{http://www.brnenskamhd.net/?p=3398}.
\item \textbf{Koncert} -- zapožičať klávesy do átria a pozvať prof.\ Slováka na koncert pre študentov.
\item \textbf{Skákací panák} -- nakresliť skákacieho panáka do vnútra za vstupné dvere FI (páskou na zem, aby sa dal bez problémov zložiť). Možno začať inde ako pred vrátnicou.
\item \textbf{Push to add...} -- veľké tlačítko na prednášku, ktoré pridáva drámu/party time/horor/... Inšpirácia známou reklamou \url{https://www.youtube.com/watch?v=316AzLYfAzw}.
\item \textbf{Alternatívne vypínače} -- pridať do prednáškovej miestnosti falošné vypínače rovnakého typu, ako tie na svetlo a označiť ich popiskami typu \textit{múdrosť zapnúť/vypnúť}, \textit{aktivita žiakov zapnúť/vypnúť}. Alebo jednoducho pridať ďalších 10 rovnakých neoznačených vypínačov.
\item \textbf{Partner do výťahu} -- postaviť do výťahu konverzačného partnera, ktorý \uv{Vás bude sprevádzať Vašou cestou}. Prípadne stolík a bude s vami hrať šach, vysvetľovať zaujímavé teorémy. Myšlienka ústnej skúšky vo výťahu (\uv{Vyberte si poschodie!}). Respektíve človek, ktorý bude podávať zaujímavé informácie (\uv{Bohužuiaľ, vaše UČO nie je prvočíselné.}).
\end{itemize}

\subsection*{Plány do blízkej budúcnosti}
\begin{itemize}[itemsep=0pt]
\item Zorganizovať niekoľko menších (spontánnych MiniFIORDov).
\item Pokúsiť sa zaviesť pravidelné stretnutia spolku s bojmi/stolnými hrami/frisbee/...
\item Preskúmať možnosti Schrödingerovho grilu.
\item Usporiadať InterJeLena pre organizátorov Soba a druhého pre celé osadenstvo spolku.
\item Zahrať si behaciu hru o polynómoch (\uv{-2.5! Faktorizujem vás!}).
\end{itemize}
\textbf{Zapísal:} Martin Ukrop\\
\textbf{Skontroloval:} Jan Mrázek

\end{document}
